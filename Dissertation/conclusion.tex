\chapter*{Заключение}                       % Заголовок
\addcontentsline{toc}{chapter}{Заключение}  % Добавляем его в оглавление

%% Согласно ГОСТ Р 7.0.11-2011:
%% 5.3.3 В заключении диссертации излагают итоги выполненного исследования, рекомендации, перспективы дальнейшей разработки темы.
%% 9.2.3 В заключении автореферата диссертации излагают итоги данного исследования, рекомендации и перспективы дальнейшей разработки темы.
%% Поэтому имеет смысл сделать эту часть общей и загрузить из одного файла в автореферат и в диссертацию:

Убегающие электроны могут играть существенную роль в некоторых процессах, происходящих в термоядерной плазме токамаков. С помощью измерений спектров жёсткого рентгеновского излучения можно восстановить функцию распределения убегающих электронов, исследовать их поведение и эволюцию во времени. Возможно определить максимальную энергию электронов, оценить переносимый ими ток. В связи с тем, что пучёк убегающих электронов может повреждать конструкционные элементы токамаков, диагностика убегающих электронов имеет большую важность на крупных установках, как существующих (JET, Asdex Upgrade), так и на проектируемых, в том числе на токамаке ИТЭР. 

Настоящая работа посвящена диагностике убегающих электронов по жёсткому рентгеновскому излучению, и охватывает как вопросы сбора данных, так и получения из них физической информации, для чего требуется обработка измеренной осциллограммы, построение спектра, восстановление по спектру функции распределения по энергии. Приведены некоторые выводы о поведении убегающих электронов в плазме токамаков.

Основные результаты работы заключаются в следующем:

%% Согласно ГОСТ Р 7.0.11-2011:
%% 5.3.3 В заключении диссертации излагают итоги выполненного исследования, рекомендации, перспективы дальнейшей разработки темы.
%% 9.2.3 В заключении автореферата диссертации излагают итоги данного исследования, рекомендации и перспективы дальнейшей разработки темы.
\begin{enumerate}
  \item были завиты методы обработки сигналов со сцинтиляционных детекторов. Разработанные и модифицированные методы обработки сигналов позволяют обрабатывать сигналы с большим количеством наложений. Тестирование методов проведено на модельных сигналах. Так, для параметров сигнала, характерных для детекторов на основе LaBr3(Ce), при загрузке $10^7$~с${}^{-1}$ ошибка определения загрузки составила 15\%, а разрешение на линии 8~МэВ при этой загрузке, обусловленное наложениями, ошибками обработки и шумами сигнала, составило 0.101~МэВ. Разработанные и модифицированные алгоритмы были реализованы в компьютерном коде ``DeGaSum''. Код успешно используется для обработки сигналов и построения спектров жёсткого рентгеновского излучения на установках Глобус-М2, Туман-3М, ФТ-2, Asdex Upgrade, JET;

  \item были модифицированы методы восстановления исходных спектров гамма и жёсткого рентгеновского излучения и функции распределения убегающих электронов по энергии по измеренным спектрам излучения. Тестирование методов проведено на сигналах от калибровочных источников и модельных сигналах. Разработанные и модифицированные алгоритмы были реализованы в компьютерном коде ``DeGaSum''. Код успешно используется для обработки спектров и восстановления функцкции распределения убегающих электронов по энергиям на установках Туман-3М, ФТ-2, Asdex Upgrade, JET;

  \item на токамаке Asdex Upgrade был разработан и введён в эксплуатацию спектрометр жёсткого рентгеновского излучения REGARDS. Спектрометр состит из детектора на основе кристалла LaBr3(Ce), сигнал с которого подаётся на плату АЦП NI~5772-02, затем сохраняется на жёсткий диск. Частота оцифровки составляет 400~МГц, время оцифровки --- 20~секунд. Для сохранения такого потка данных на управляющий компьютер были использованы алгоритмы сжатия данных в реальном времени без потерь, реализованные в компьютером коде ``DeGaSum''. Для управления спектрометром (запуск, сохранение и обработка данных) использовался компьютерный код ``DeGaSum'';

  \item с помощью разработанного спектрометра REGARDS и имевшегося на токамаке спектрометра AUG-HXR было исследовано поведение убегающих электронов в токамаке Asdex Upgrade. Была проведена обработка сигналов с помощью кода ``DeGaSum'', получены функции распределения убегающих электронов по энергии, максимальня энергия убегающих электронов. Исследована эволюция убегающих электронов от времени. Показано, что с увеличением количества инжектируемого аргона ток убегающих электронов растёт линейно. Максимальная энергия убегающих электронов слабо зависит от количества инжектируемого аргона, спадая с ростом его концентрации;

  \item были проведены измерения жёсткого рентгеновского излучения на токамаке JET с помощью спектрометров на основе кристаллов NaI(Tl) с вертикальной линией обзора. Была проведена обработка спектров излучения с помощью кода ``DeGaSum'', получены функции распределения убегающих электронов по энергии. Исследована эволюция убегающих электронов от времени.

\end{enumerate}


Таким образом, перечисленные выше результаты позволяют сделать вывод об успешном решении поставленных задач и достижения цели исследования.

В заключение автор выражает благодарность научному руководителю А.~Е.~Шевелеву за научное руководство при написании диссертации, а так же в научной деятельности автора диссертации, М.~В.~Ильясовоу за сотрудничество и помощь в публикации многих полученных результатов, В.~Г.~Киптилому за помощь в вопросах взаимодействия и работы на токамаке JET.

