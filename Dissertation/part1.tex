\chapter{Генерация и транспорт убегающих электронов в плазме токамаков, а так же использование сцинтиляционных детекторов для их диагностики}\label{ch:ch1}

\section{Механизмы генерации убегающих электронов}\label{sec:ch1/sec1}

В плазме токамаков имеют место несколько механизмов генерации убегающих электронов. Это:

\begin{itemize}
  \item <<Традиционный>> механизм генерации убегающих электронов в поле Дрейсера \cite{Golant1977,Dreicer1959};
  \item Лавинный механизм, при котором первичные убегающие электроны при взаимодействии с электронами основной плазмы перевозят их в режим убегания;
  \item Прочие механизмы, такие как генерация убегающих электронов при распаде трития или при комптоновском рассеянии гамма излучения 
\end{itemize}

Рассмотрим их по порядку.

\subsection{Убегание электронов в сильных электрических полях в плазме}

Эффект убегания был предсказан в 1925 году \cite{Wilson1925} и развит в \cite{Dreicer1959}. Пусть в сильноионизованной плазме имеется электрическое поле $E$. Тогда для перехода в режим непрерывного ускорения сила, действующя на электрон со стороны электрического поля, должна превышать силу трения, вызванную кулоновскими столкновениями в плазме:

\begin{equation}
  \label{eq:runawayEq}
  e E >  m_e v_e \nu_{ei}(v_e)
\end{equation}
где $e$ и $m_e$ --- заряд и масса электрона, $\nu_{e}$ --- частота столкновений электрона, $v$ --- скорость электрона \cite{Wesson2004}. Частота кулоновских электрон-ионных столкновений зависит от скорости. Для случая, когда $v >> \sqrt{ 2 T_e / m_e }$, где $T_e$ --- средняя температура электронов, частота электронных столкновений может быть представлена как 
\begin{equation}
  \nu_{e}(v) = \frac{ 3 e^4 n \Lambda }{ 4 \pi \epsilon_0^2 m_e^2 v^3 }
\end{equation} 
где $\Lambda$ --- кулоновский логарифм, $n$ --- концентрация плазмы \cite{Wesson2004}. Можно заметить, что существует некая критическая скорость, начиная с которой неравенство \ref{eq:runawayEq} оказывается истинным. Эта критическая скорость равна
\begin{equation}
  \label{eq:critVelocity}
  v_c = \sqrt{ \frac{ 3 n e^3 \Lambda }{ 4 \pi \epsilon_0^2 m_e E } }
\end{equation}
Если каким-то образом образуется электрон со скоростью, больше кртитической скорости $v_c$, то он переходит в режим неограниченного ускорения. Такие электроны называются <<убегающими>> \cite{Golant1977}.

Можно рассчитать величину электрического поля, при котором в режим убегания переходят электроны с тепловой скоростью. Это поле оказывается равным 
\begin{equation}
  E_D = \frac{ n e^3 \Lambda }{ 4 \pi \epsilon_0^2 m_e v_{Te}^2 }
\end{equation}
и называется полем Дрейсера \cite{Dreicer1959,Golant1977,Wesson2004}. 

Когда электрическое поле достаточно мало, критическая скорость, рассчитанная по уравнению \ref{eq:critVelocity}, приближается к скорости света. Для релятивистских электронов время замедления почти постоянно и существенного уменьшения силы трения с увеличением их энергии уже не происходит. Соответственно, для электрических полей, таких что
\begin{equation}
  E < \frac{ n e^3 \Lambda }{ 4 \pi \epsilon_0^2 m_e c^2 }
\end{equation}
генерации убегающих электронов не происходит ни при каких обстоятельствах.

\section{Механизмы ограничения энергии убегающих электронов}


В случае, когда в плазме длина свободного проб


\section{Методы диагностики убегающих электронов}\label{sec:ch1/sec2}

МОИ ССЫЛКИ: \cite{Khilkevitch2020} \cite{Reux2015}

\FloatBarrier
